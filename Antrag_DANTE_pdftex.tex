\documentclass[ngerman]{article}
\usepackage[T1]{fontenc}
\usepackage{libertine}
\usepackage{babel}
\usepackage[utf8]{inputenc}
\usepackage{comment}
%\begin{comment}
\usepackage[%
    web={pro,
        tight,
        german,
        nodirectory,
        usesf
        },
    eforms,
    aebxmp,
    attachsource={tex}
]{aeb_pro}

\usepackage{dtklogos}


\usepackage{calc}
\usepackage{eforms}

\screensize{297mm}{210mm} % height, width
\margins{.6in}{.6in}{.5in}{.5in}% left,right,top, bottom


%% Fu{\ss}note in Tabelle
\newcounter{myfootertablecounter}
\newcommand\myfootnotemark{%
\addtocounter{footnote}{1}%
\footnotemark[\thefootnote]%
}%
\newcommand\myfootnotetext[1]{%
\addtocounter{myfootertablecounter}{1}
\footnotetext[\value{myfootertablecounter}]{#1}
}

%% Layoutparameter
\parindent0pt
\parskip3pt
\pagestyle{empty}

\begin{document}

\hfill{\LARGE{Antrag zur Mitgliedschaft bei DANTE e.V.}}

\pushButton[\S{S}\BC{}\BG{}\textColor{1 0 0 rg}\F{-\FPrint}\H{N}\CA{Formular bitte am Rechner ausf\string\374llen und dann ausdrucken.}]{name}{\textwidth}{12bp}

\fcolorbox{lightgray}{lightgray}{\parbox{\linewidth}{\textit{Name mit Anschrift, gewünschte Mitgliedsgruppe und Unterschrift müssen im Folgenden auf jeden Fall angegeben werden, damit der Antrag bearbeitet werden kann.}}}

\vspace{6pt}

\newcounter{infoLineNum}
\setcounter{infoLineNum}{0}
\newcommand{\infoInput}[2][5.5in]{\stepcounter{infoLineNum}%
    \makebox[0pt][l]{\kern4pt\raisebox{.75ex}{\textField[\W0\BC{}\BG{}\TU{#2}]{name\theinfoLineNum}{#1}{12bp}}}\dotfill}

\begin{tabular}{lp{5.5in}}
    Nachname, Vorname:\myfootnotemark   & \infoInput{Nachname, Vorname}\\[6pt]
    Institut/Firma:                     & \infoInput{Institut/Firma}\\[6pt]
    Stra{\ss}e:                             & \infoInput{Strasse}\\[6pt]
    PLZ Ort:                            & \infoInput{PLZ Ort}\\[6pt]
    Telefon:                            & \infoInput{Telefon}\\[6pt]
    Fax:                                & \infoInput{Fax}\\[6pt]
    E-Mail:                             & \infoInput{E-Mail}\\[6pt]
    WWW:                                & \infoInput{WWW}
\end{tabular}

\myfootnotetext{Von Firmen, Institutionen des öffentlichen Rechts und Forschungseinrichtungen ist hier der Name der vertretungsberechtigten
Person anzugeben. Bitte immer nur einen Namen pro Mitgliedschaft nennen.}

\newcounter{checkboxcounter}
\setcounter{checkboxcounter}{0}
\def\firstCk{\stepcounter{checkboxcounter}\checkBox{checkbox\thecheckboxcounter}{11bp}{11bp}{ja}}
\renewcommand{\labelitemi}{\firstCk}
\begin{itemize}
\item Ich will mit der oben genannten E-Mail-Adresse in die vereinsinterne und geschlossene Kommunikationsliste\\
\texttt{dante-ev@dante.de} aufgenommen werden.
\item  Ich möchte, dass oben stehende Angaben in die Mitgliederliste aufgenommen werden. Diese Liste wird ausschließlich
an Mitglieder verteilt.
\item  Ich war schon Mitglied des Vereins mit der Mitgliedsnummer \makebox[0.75in][l]{\infoInput[0.75in]{Mitgliedsnummer}}\hfill
\item  Ich habe auf folgendem Weg von DANTE e.V. erfahren: \infoInput[1.75in]{DANTE ist mir schon bekannt durch}
\end{itemize}
Ich ordne mich einer der folgenden Mitgliedsgruppen zu:
\begin{itemize}
\item Firma, die Produkte in Verbindung mit \TeX{} anbietet: 150 €
\item Firma, die \TeX{} anwendet: 150 €
\item Institution des öffentlichen Rechts oder Forschungseinrichtung: 65 €
\item Privatperson: 40 €
\item Privatperson mit Ermäßigung:\myfootnotemark\, 20 € (bitte Bescheinigung beilegen)
\item Schüler/in mit Ermäßigung: 15 € (bitte Bescheinigung beilegen)
\end{itemize}
\myfootnotetext{Eine solche Ermäßigung gibt es für Studierende, Rentner/innen, Arbeitslose, Zivil- und Wehrdienstleistende. Senden Sie bitte eine entsprechende Bescheinigung mit, da ohne sie eine nachträgliche Hochstufung in die Gruppe \textit{Privatperson} erfolgen muss.}
Die Mitgliedsbeiträge gelten für das laufende Kalenderjahr (\textit{1. Januar bis 31. Dezember}). Zahlungsaufforderungen
werden aus Kosten- und Zeitgründen nur auf explizite Anfrage hin verschickt, während Sie eine Quittung automatisch
erhalten. Zahlungsmodalitäten siehe Rückseite.
\begin{itemize}
\item Der Beitrag wird überwiesen.
\item Bitte schicken Sie mir eine Zahlungsaufforderung.
\item Eine unterschriebene Einzugsermächtigung liegt bei.
\end{itemize}

Senden Sie den ausgefüllten Antrag an:
\begin{quote}
DANTE

Deutschsprachige Anwendervereinigung \TeX{} e.V.

Postfach 10 18 40

69008 Heidelberg

Deutschland
\end{quote}
Alle für das laufende Kalenderjahr bereits versandten Ausgaben der Vereinszeitung und eventuell weitere Leistungen
wie DVD/CD-ROMs sind (solange vorrätig) im Beitrag enthalten. Die Mitgliedschaft tritt erst nach Eingang des
eigenhändig unterschriebenen Antrages und des Mitgliedsbeitrags bei DANTE e.V. in Kraft und wird durch die Zusendung
von Mitgliedsnummer und Satzung bestätigt. Alle Dienstleistungen des Vereins können erst ab Inkrafttreten
der Mitgliedschaft in Anspruch genommen werden.

\vspace{22pt}

\begin{minipage}{0.2\textwidth}
\begin{center}
\dotfill

Ort
\end{center}
\end{minipage}\hfill
\begin{minipage}{0.15\textwidth}
\begin{center}
\dotfill

Datum
\end{center}
\end{minipage}\hfill
\begin{minipage}{0.5\textwidth}
\begin{center}
\dotfill

Unterschrift
\end{center}
\end{minipage}


%% Rückseite
\newpage

\renewcommand\thesubsection{\arabic{subsection}.}

\begin{flushright}
{\Huge DANTE} \hfill Deutschsprachige Anwendervereinigung \TeX\ e.V.\\
Bergheimer Straße 110a\\
69115 Heidelberg
\end{flushright}

\bigskip
Gläubiger-Identifikationsnummer DE31ZZZ00000343567\\[5pt]
Mandatsreferenz: Mgl. \# %\textbf{Wird separat von DANTE \eV\ mitgeteilt}
\\[1cm]

\textbf{\textsf{V\,E\,R\,T\,R\,A\,G}}

\bigskip
Vorname und Name (Kontoinhaber)\\
\makebox[12cm]{\infoInput[12cm]{Vorname und Name (Kontoinhaber)}}

\bigskip
Straße und Hausnummer\\
\makebox[12cm]{\infoInput[12cm]{Straße und Hausnummer}}

\bigskip
Postleitzahl und Ort\\
\makebox[12cm]{\infoInput[12cm]{Postleitzahl und Ort}}


\bigskip
Datum, Ort und Unterschrift\\
\makebox[12cm]{\infoInput[12cm]{Datum, Ort und Unterschrift}}

\bigskip
\textbf{Erteilung einer Einzugsermächtigung und eines SEPA-Lastschriftmandats}


\subsection{Einzugsermächtigung}
Ich ermächtige DANTE \eV\ widerruflich, die von mir zu entrichtenden Zahlungen bei Fälligkeit 
durch Lastschrift von meinem Konto einzuziehen.

\subsection{SEPA-Lastschriftmandat}

Ich ermächtige DANTE \eV, Zahlungen von meinem Konto mittels Lastschrift einzuziehen. 
Zugleich weise ich mein Kreditinstitut an, die von DANTE \eV\ auf mein Konto 
gezogenen Lastschriften einzulösen.


\emph{Hinweis:} Ich kann innerhalb von acht Wochen, beginnend mit dem Belastungsdatum,
die Erstattung des belasteten Betrages verlangen. Es gelten dabei die mit meinem
Kreditinstitut vereinbarten Bedingungen.


\bigskip
Kreditinstitut (Name und BIC)\\
\makebox[15cm]{\infoInput[15cm]{Kreditinstitut (Name und BIC)}}


\def\VierU{\string\137 \string\137 \string\137 \string\137 }
\bigskip
%\resizebox{10cm}{!}{
\ttfamily
\infoInput[4em]{DE \string\137 \string\137}|%
\infoInput[8em]{\VierU \VierU \VierU \VierU }\infoInput[2em]{\kern2pt\_\kern2pt\_}\\
%\resizebox{10cm}{!}{\texttt{DE\kern2pt\_\kern2pt\_|\VierU\VierU\VierU\VierU\kern2pt\_\kern2pt\_}}\\
%\resizebox{10cm}{!}{\texttt{DE\_\_|\_\_\_\_|\_\_\_\_|\_\_\_\_|\_\_\_\_|\_\_}}\\
\makebox[10cm]{IBAN}


\rmfamily
Vor dem ersten Einzug einer SEPA-Basislastschrift wird mich DANTE \eV\ über
den Einzug in dieser Verfahrensart unterrichten. 
\end{document} 
