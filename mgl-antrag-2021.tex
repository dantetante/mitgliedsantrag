\documentclass[12pt,ngerman,parskip]{scrartcl}
%\documentclass[8pt,ngerman,parskip]{scrartcl}
\usepackage[margin=1cm]{geometry}
\usepackage{libertinus}
\usepackage[right]{eurosym}
\usepackage{babel}
\usepackage{setspace}
\usepackage[inline]{enumitem}
\usepackage{tabularx}
\usepackage{footnote}
\usepackage{tcolorbox}
\usepackage{xcolor}
\usepackage{tikz}

\pagestyle{empty}

\usepackage{hyperref}

\begin{document}

\Large\hfill Antrag zur Mitgliedschaft bei DANTE e.\,V.

\normalsize

\begin{tcolorbox}[sharp corners=all]
\textit{\small Die mit einem Stern gekennzeichneten Angaben sind
zwingend erforderlich,
damit der Antrag bearbeitet werden kann.}
\end{tcolorbox}


\begin{savenotes}
\begin{Form}

%\renewcommand\LayoutTextField[2]{%
%\raisebox{-\dp\strutbox}{\makebox[0pt][l]{\cleaders
%\hbox to .44em{\hss.\hss}\hskip\csname Fld@width\endcsname}}#2}
%s. https://tex.stackexchange.com/questions/398905/typesetting-fillable-pdf-forms-with-hyperref

 \renewcommand\LayoutTextField[2]{%
   \raisebox{-3.2pt}{\tikz[overlay]\draw[dashed](0,-0.2\dp\strutbox)--++(\csname Fld@width\endcsname,0pt);#2}}


\begin{spacing}{1.5}
\begin{tabularx}{\textwidth}{lX}
*Nachname, *Vorname:\footnote{%
Von Firmen, Institutionen des \"{o}ffentlichen Rechts und Forschungseinrichtungen
ist hier der Name der vertretungsberechtigten
Person anzugeben. Bitte immer nur einen Namen pro Mitgliedschaft nennen.}
&
\TextField[bordercolor={gray},
backgroundcolor={},
width=5in,]{}\\
Institut/Firma:
&
\TextField[bordercolor={gray},
width=5in,]{}\\
*Straße:&
\TextField[bordercolor={gray},
backgroundcolor={},
width=5in,]{}\\
{\tiny evtl. weitere Adressdaten:}&
\TextField[bordercolor={gray},
backgroundcolor={},
width=5in,]{}\\
*PLZ *Ort:&
\TextField[bordercolor={gray},
width=5in,]{}\\
Land (falls nicht Deutschland):&
\TextField[bordercolor={gray},
width=5in,]{}\\
E-Mail:
&
\TextField[bordercolor={gray},
width=5in,]{}\\
\end{tabularx}


\end{spacing}
\def\LayoutCheckField#1#2{#2\quad #1}




Die Mitgliedsbeitr"age bei Dante h"angen von der
Mitgliedsart ab:
\begin{itemize*}
\item
Firma: \EUR{150}
\item Institut des "offentlichen Rechts oder
Forschungseinrichtung: \EUR{65}
\item Privatperson: \EUR{40}
\item Privatperson ermäßigt: \EUR{20}
\item Schüler:in: \EUR{15}
\end{itemize*}\\
Eine Erm"a"sigung gibt es f"ur Studierende, Renter:innen,
Arbeitslose sowie Personen im Bundesfreiwilligendienst. Eine
Bescheinigung hierf"ur ist -- je nach Erm"a"sigungstyp --
einmalig oder aber
jedes Jahr erneut vorzulegen. Sch"uler:innen m"ussen
einen Sch"ulerausweis o.\,"a. vorlegen.
Die Bescheinigung wird f"ur das gesamte aktuelle Kalenderjahr
akzeptiert, auch wenn die Bescheinigung nur f"ur einen Teil
dieses Jahres g"ultig ist.
Bei fehlender Bescheinigung erfolgt eine nachtr"agliche
Hochstufung in die Gruppe \emph{Privatperson}.

\ChoiceMenu[combo]{%
*Ich ordne mich der folgenden
Mitgliedsgruppe zu:
}{Firma,{Institut, Forschungseinrichtung}, Privatperson,
Privatperson ermäßigt, Schüler:in}

Die Mitgliedsbeitr"age gelten f"ur das laufende Kalenderjahr
(1. Januar bis 31. Dezember). Sie erhalten jedes Jahr zu
Jahresbeginn eine
Rechnung "uber den Mitgliedsbeitrag f"ur das aktuelle Jahr und
eine Zuwendungsbescheinigung f"ur das vergangene Jahr,
normalerweise per Post. Abweichend hiervon k"onnen Sie
w"ahlen:\\
\CheckBox[]{%
Ich m"ochte die Rechnung und Zuwendungsbescheinigung nur per E-Mail
erhalten.}\\
\CheckBox[]{%
Ich m"ochte die Rechnung und Zuwendungsbescheinigung
zus"atzlich
auch per E-Mail erhalten.}


Den Mitgliedsbeitrag k"onnen Sie entweder selbst "uberweisen
oder uns ein SEPA-Lastschriftmandat erteilen:\\
\CheckBox[]{%
Ich werde den Beitrag jeweils nach Erhalt der Rechnung
selbst "uberweisen.}\\
\CheckBox[]{%
Ich m"ochte am Lastschriftverfahren teilnehmen. Die
Einzugserm"achtigung (zu finden hier:
\url{urlwodassepadingsliegenwird}) habe ich ausgef"ullt,
h"andisch unterschrieben
und per Post an das Danteb"uro geschickt.}

Alle f"ur das laufende Kalenderjahr bereits versandten
Ausgaben der Vereinszeitung und eventuell weitere Leistungen
(z.\,B. der Datentr"ager der \TeX-Collection, zur Zeit noch eine
DVD) sind im Beitrag enthalten. Die Mitgliedschaft tritt
erst nach Eingang des eigenh"andig unterschriebenen Antrages
und des Mitgliedsbeitrags bei Dante e.\,V. in Kraft und wird
durch Zusendung von Mitgliedsnummer und Satzung best"atigt.
Alle Dienstleistungen des Vereins k"onnen erst ab
Inktafttreten der Mitgliedschaft in Anspruch genommen
werden.

\CheckBox[]{*Die Satzung (\url{linkzursatzung}) habe ich
gelesen und erkenne sie an.}

\CheckBox[]{*Ich habe die Datenschutzerkl"arung (siehe
unten) gelesen und
akzeptiere sie.}

\bigskip

*Ort, *Datum: \TextField[bordercolor={gray}, width=.35\textwidth,]{}
*Unterschrift: \underline{\hspace*{.35\textwidth}}

\bigskip

Senden Sie den ausgef"ullten Antrag an:
Dante e.\,V.,
Postfach 11\,03\,61,
69072 Heidelberg,
Deutschland\\

\bigskip

Ich habe auf folgendem Weg von Dante e.\,V. erfahren:\\
\TextField[%bordercolor={},
width=\textwidth,]
{}\\

\CheckBox{%
%\CheckBox[multiline=true]{%
Ich m"ochte mit der von mir angegebenen E-Mail-Adresse in die
vereinsinterne und geschlossene Kommunikationsliste
\texttt{dante-ev@dante.de} aufgenommen werden.}

\end{Form}
\end{savenotes}

\vfill

\begin{spacing}{1.2}
\begin{tcolorbox}
[sharp corners=all,
title={\Large Informationen zum Datenschutz
entsprechend Art. 13 DSGVO}]
\begin{itemize}[nosep]
\item Verantwortlich ist:\\
Dante e.\,V., Bergheimerstr. 110A, 69115 Heidelberg;
Vertreten durch den Vorstand; Kontakt:
\mbox{\texttt{office@dante.de}}; Ein
Datenschutzbeauftragter ist f"ur den Verein nicht bestellt.
\item Zweck der Verarbeitung ist die ordentliche Verwaltung der
Vereinsmitglieder. Die Rechtsgrundlage f"ur die Verarbeitung
ergibt sich aus Art. 6 Abs. 1 lit. b DSGVO.
\item Die Daten werden vom Vorstand und von mit der Verwaltung
beauftragten Personen im Sekretariat des Vereins
verarbeitet. Daneben k"onnen die
Rechnungspr"ufer:innen w"ahrend der Rechnungspr"ufung Einblicke in
die Beitrags- und Finanzverwaltung erhalten.
\item
Steuerrelevante Informationen werden zehn Jahre aufbewahrt.
Entsprechend der Handelsgesetzgebung heben wir
Mitgliedsantr"age und Stammdaten sechs Jahre nach Austritt
des Mitglieds auf.
\item
Sie haben das Recht auf Auskunft "uber die betreffenden
personenbezogenen Daten. Dar"uber hinaus haben Sie das Recht
auf Berichtigung und Daten"ubertragbarkeit. Auf Grund der
gesetzlich vorgeschriebenen Aufbewahrungsfristen k"onnen wir
Ihre Daten nicht vorzeitig l"oschen. Nach Beendigung der
Mitgliedschaft werden die Daten archiviert und nicht weiter
verarbeitet.
\item
Sie haben das Recht, sich bei einer Aufsichtsbeh"orde zu
beschweren.
\end{itemize}
\end{tcolorbox}
\end{spacing}

\vfill
\tiny
Version vom \today
\end{document}




