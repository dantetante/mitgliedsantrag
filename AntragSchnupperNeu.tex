\documentclass[ngerman,dvips]{article}
\usepackage[T1]{fontenc}
\usepackage{libertine}
\usepackage{babel}
\usepackage[utf8]{inputenc}
\usepackage{comment}
%\begin{comment}
\usepackage[%
    driver=dvips,
    web={pro,
        tight,
        german,
        nodirectory,
        usesf
        },
    eforms,
    aebxmp,
    attachsource={tex}
]{aeb_pro}


\DeclareDocInfo{
    title=Antragsformular DANTE e.V.,
    author=DANTE e.V.,
    university=,
    email=herbert@dante.de,
    subject=,
    keywords={Formularfelder mit eForms},
    talksite=,
    talkdate={March, 2014},
    copyrightStatus=True,
    copyrightNotice={Copyright (C) \the\year, DANTE e.V.},
    copyrightInfoURL=http://www.dante.de
}

\DeclareInitView
{%
layoutmag={mag=FitWidth},%
windowoptions={showtitle},
}

\begin{docassembly}
\executeSave()
\end{docassembly}
%\end{comment}
%\def\DANTE{{\fontfamily{dante}\selectfont DANTE}}
\usepackage{dtklogos}

%\begin{comment}

\usepackage{calc}
%\usepackage[tight,
%            german,
%            nodirectory,
%            usesf,
%            pdftex
%            ]{web}
\usepackage[dvips]{eforms}

    \title{Antragsformular DANTE e.V.}
    \author{DANTE e.V.}
    \keywords{Formularfelder mit eForms}
%\end{comment}



\screensize{297mm}{210mm} % height, width
\margins{.6in}{.6in}{.5in}{.5in}% left,right,top, bottom


\usepackage{multicol}

%% Definition des Eurosymbols
\newcommand\eur{%
  {\sffamily C\makebox[0pt][l]{\kern-.70em\mbox{--}}%
  \makebox[0pt][l]{\kern-.68em\raisebox{.25ex}{--}}\,}}

%% Fu{\ss}note in Tabelle
\newcounter{myfootertablecounter}
\newcommand\myfootnotemark{%
\addtocounter{footnote}{1}%
\footnotemark[\thefootnote]%
}%
\newcommand\myfootnotetext[1]{%
\addtocounter{myfootertablecounter}{1}
\footnotetext[\value{myfootertablecounter}]{#1}
}

\usepackage{dtklogos}
%% Layoutparameter
\parindent0pt
\parskip3pt
\pagestyle{empty}

\begin{document}

\hfill{\LARGE{Antrag zur Mitgliedschaft bei \dante}}

\pushButton[\S{S}\BC{}\BG{}\textColor{1 0 0 rg}\F{-\FPrint}\H{N}\CA{Formular am Rechner ausf\string\374llen:
Nutzen Sie die 'tab' Funktion Ihrer Tastatur und drucken Sie das Formular dann aus.}]{name}{\textwidth}{12bp}

\fcolorbox{lightgray}{lightgray}{\parbox{\linewidth}{\itshape Die Schnuppermitgliedschaft ist eine günstige 
Möglichkeit, \dante\ zum Preis von 15 € kennenzulernen. Der
Beitrag für die Schnuppermitgliedschaft gilt für das laufende Kalenderjahr (1. Januar bis 31. Dezember), die
Mitgliedschaft \textbf{endet automatisch} mit Ablauf dieses Kalenderjahres.}}

\vspace{6pt}

\newcounter{infoLineNum}
\setcounter{infoLineNum}{0}
\newcommand{\infoInput}[2][5.5in]{\stepcounter{infoLineNum}%
    \makebox[0pt][l]{\kern4pt\raisebox{.75ex}{\textField[\W0\BC{}\BG{}\TU{#2}]{name\theinfoLineNum}{#1}{12bp}}}\dotfill}

\begin{tabular}{lp{5.5in}}
    Nachname, Vorname:\myfootnotemark   & \infoInput{Nachname, Vorname}\\[6pt]
    Institut/Firma:                     & \infoInput{Institut/Firma}\\[6pt]
    Stra{\ss}e:                             & \infoInput{Strasse}\\[6pt]
    PLZ Ort:                            & \infoInput{PLZ Ort}\\[6pt]
    Telefon:                            & \infoInput{Telefon}\\[6pt]
    Fax:                                & \infoInput{Fax}\\[6pt]
    E-Mail:                             & \infoInput{E-Mail}\\[6pt]
    WWW:                                & \infoInput{WWW}
\end{tabular}

\myfootnotetext{Von Firmen, Institutionen des \"{o}ffentlichen Rechts und Forschungseinrichtungen ist hier der Name der vertretungsberechtigten
Person anzugeben. Bitte immer nur einen Namen pro Mitgliedschaft nennen.}


\vspace{2cm}
Ich habe auf folgendem Weg von DANTE e.V. erfahren: \infoInput[1.75in]{DANTE ist mir schon bekannt durch}


\bigskip
\newcounter{checkboxcounter}
\setcounter{checkboxcounter}{0}
\def\firstCk{\stepcounter{checkboxcounter}\checkBox{checkbox\thecheckboxcounter}{11bp}{11bp}{ja}}
\renewcommand{\labelitemi}{\firstCk}
\begin{itemize}
\item Der Beitrag wird \"{u}berwiesen.
\item Der Beitrag wurde bereits bar entrichtet.
\end{itemize}



\vspace{1cm}
Die Zahlungsmodalitäten sind auf der Rückseite beschrieben. Senden Sie den ausgefüllten Antrag an:

\vspace{1cm}
DANTE, Deutschsprachige Anwendervereinigung TEX e.\kern-1pt V.\\
Postfach 10 18 40\\
69008 Heidelberg\\
Deutschland



\vspace{1cm}
Alle f\"{u}r das laufende Kalenderjahr bereits versandten Ausgaben der Vereinszeitung und eventuell weitere Leistungen
wie DVD/CD-ROMs sind (solange vorr\"{a}tig) im Beitrag enthalten. Die Mitgliedschaft tritt erst nach Eingang des
eigenh\"{a}ndig unterschriebenen Antrages und des Mitgliedsbeitrags bei DANTE e.V. in Kraft und wird durch die Zusendung
von Mitgliedsnummer und Satzung best\"{a}tigt. Alle Dienstleistungen des Vereins k\"{o}nnen erst ab Inkrafttreten
der Mitgliedschaft in Anspruch genommen werden.


\vspace{22pt}

\begin{minipage}{0.2\textwidth}
\begin{center}
\dotfill

Ort
\end{center}
\end{minipage}\hfill
\begin{minipage}{0.15\textwidth}
\begin{center}
\dotfill

Datum
\end{center}
\end{minipage}\hfill
\begin{minipage}{0.5\textwidth}
\begin{center}
\dotfill

Unterschrift
\end{center}
\end{minipage}


%% Rückseite
\newpage

\renewcommand\thesubsection{\arabic{subsection}.}

\begin{flushright}
{\Huge\DANTE} \hfill Deutschsprachige Anwendervereinigung \TeX\ \eV\\
Bergheimer Straße 110a\\
69115 Heidelberg
\end{flushright}


Folgende Bankverbindung steht zur Verfügung:

Für Überweisungen aus allen Ländern:\\
Volksbank Rhein-Neckar eG\\
Kontonummer: 2 310 007\\
Bankleitzahl: 670 900 00\\
Swift: GENODE61MA2\\
IBAN: DE67 6709 0000 0002 3100 07



\end{document} 